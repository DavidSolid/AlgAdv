\documentclass[]{article}


% Title Page
\title{Homework1, Algoritmi su Grafi}
\author{Enrico Cancelli, Alessandro Pegoraro}


\begin{document}
\maketitle

\begin{abstract}
	Una breve relazione sullo svolgimento del progetto. La relazione deve contenere:
	\begin{itemize}
		\item una sezione introduttiva con la descrizione degli algoritmi e delle scelte implementative che avete fatto;
		\item grafici esplicativi dei risultati con le risposte alle due domande;
		\item eventuali originalità introdotte nell'elaborato o nell'implementazione;
		\item una sezione conclusiva in cui porre i vostri commenti e le vostre conclusioni sull’elaborato svolto e i risultati ottenuti
	\end{itemize}
	\textit{/*da eliminare*/}
\end{abstract}

\section{Introduzione}
Lo scopo di questo progetto è l'implementazione e l'analisi di 3 algoritmi di ricerca per il \textit{Minimum Spanning Tree} (in seguito solo MST) di un grafo pesato e non diretto.\\
Questi algoritmi sono:
\begin{itemize}
	\item: Algoritmo di Prim: [...] 
	\item: Algoritmo di Kruskal (implementazione naive): [...]
	\item: Algoritmo di Kruskal con struttura dati \textit{Union-Find}: [...]
\end{itemize}
Nelle successive sezioni descriveremo in dettaglio l'implementazione di questi algoritmi e delle relative strutture dati di supporto e li testeremo su un dataset generato randomicamente, analizzando i risultati ottenuti e le performance in termini di tempo d'esecuzione e spazio allocato in memoria.\\
I test saranno condotti su grafi non necessariamente semplici (quindi con la possibile presenza di \textit{self loops} e archi multipli tra due nodi) e con pesi non necessariamente positivi.\\
Abbiamo scelto di implementare tali algoritmi utilizzando il linguaggio \textit{C++17} per motivi di efficienza e per potere, allo stesso tempo, utilizzare astrazioni tipiche dei linguaggi ad oggetti rendendo il codice prodotto più modulare.
\section{Implementazione}
In questa sezione verranno esposte e adeguatamente motivate le scelte implementative adottate durante lo sviluppo. L'intero progetto è stato realizzato facendo il più possibile uso di codice generico (template di classe per le strutture dati di supporto e template di funzione per gli algoritmi).\\
Infine verrà data una spiegazione dettagliata sulla struttura del codice realizzato ed eventuali note per la compilazione.
\subsection{Parser}
\subsection{Strutture dati}
\subsubsection{MinHeap}
\subsubsection{Union-Find}
\subsection{Strutture per la rappresentazione di grafi}
\subsubsection{Adjacency List}
\subsection{Algoritmi}
\subsubsection{Prim}
\subsubsection{Naive Kruskal}
\subsubsection{Kruskal (con Union-Find)}
\subsection{Note su compilazione e struttura del codice}
\textit{/*forse spostare in sezione a parte?*/}
\section{Testing e analisi sperimentali}
\subsection{Risultati prodotti}
[...]
\section{Conclusioni}
\end{document}          
