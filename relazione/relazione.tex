\documentclass[]{article}
\usepackage[a4paper, total={6in, 10.5in}]{geometry}
\usepackage[ruled,vlined,linesnumbered]{algorithm2e}
\usepackage{listings}
\usepackage{xcolor}

\definecolor{codegreen}{rgb}{0,0.6,0}
\definecolor{codegray}{rgb}{0.5,0.5,0.5}
\definecolor{codepurple}{rgb}{0.58,0,0.82}
\definecolor{backcolour}{rgb}{0.95,0.95,0.92}

\lstdefinestyle{mystyle}{
    backgroundcolor=\color{backcolour},   
    commentstyle=\color{codegreen},
    keywordstyle=\color{magenta},
    numberstyle=\tiny\color{codegray},
    stringstyle=\color{codepurple},
    basicstyle=\ttfamily\footnotesize,
    breakatwhitespace=false,         
    breaklines=true,                 
    captionpos=b,                    
    keepspaces=true,                 
    numbers=left,                    
    numbersep=5pt,                  
    showspaces=false,                
    showstringspaces=false,
    showtabs=false,                  
    tabsize=2,
    frame=single
}

% Title Page
\title{Homework1, Algoritmi su Grafi}
\author{Enrico Cancelli, Alessandro Pegoraro}


\begin{document}
\maketitle

\begin{abstract}
	Una breve relazione sullo svolgimento del progetto. La relazione deve contenere:
	\begin{itemize}
		\item una sezione introduttiva con la descrizione degli algoritmi e delle scelte implementative che avete fatto;
		\item grafici esplicativi dei risultati con le risposte alle due domande;
		\item eventuali originalità introdotte nell'elaborato o nell'implementazione;
		\item una sezione conclusiva in cui porre i vostri commenti e le vostre conclusioni sull'elaborato svolto e i risultati ottenuti
	\end{itemize}
	\textit{/*da eliminare*/}
\end{abstract}

\section{Introduzione}
Lo scopo di questo progetto è l'implementazione e l'analisi di 3 algoritmi di ricerca per il \textit{Minimum Spanning Tree} (in seguito solo MST) di un grafo pesato e non diretto.\\
Questi algoritmi sono:
\begin{itemize}
	\item Algoritmo di Prim
	\item Algoritmo di Kruskal (implementazione naive)
	\item Algoritmo di Kruskal con struttura dati \textit{Union-Find}
\end{itemize}
\subsection{Pseudocodice}
Come riferimento per l'implementazione degli algoritmi è stato preso il seguente pseudo codice spiegato a lezione:\\
\begin{algorithm}[H]
	\SetAlgoLined
	\DontPrintSemicolon
	\KwIn{Graph G}
	\KwIn{vertex s}
	\KwResult{Graph of a MST}
	\For{each u $\in$ V}{
		key[u] $\gets$ +$\infty$\;
		$\pi$[u] $\gets$ NULL\;
	}
	key[s] $\gets$ 0\;
	Q $\gets$ V\;
	\While{Q $\neq$ Ø}{
		u $\gets$ extractMin(Q)\;
		\For{each v adjacent to u}{
			\If{v $\in$ Q and w(u,v) $<$ key[v]}{
				$\pi$[v] $\gets$ u\;
				key[v] $\gets$ w(u,v)\;
			}
		}
	}
	return A\;
	\caption{Prim}
\end{algorithm}

\begin{algorithm}[H]
	\SetAlgoLined
	\DontPrintSemicolon
	\KwIn{Graph G}
	\KwResult{Graph of a MST}
	A = Ø\;
	sort edges of G by cost\;
	\For{each edge e, in nondecreasing order of cost}{
		\If{A $\cup$ {e} is acyclic}{
			A = A $\cup$ {e}\;
		}
	}
	return A\;
	\caption{Kruskal Naive}
\end{algorithm}

\begin{algorithm}[H]
	\SetAlgoLined
	\DontPrintSemicolon
	\KwIn{Graph G}
	\KwResult{Graph of a MST}
	A = Ø\;
	U = initialize(V)\;
	sort edges of E by cost\;
	\For{each edge e=(v,w), in nondecreasing order of cost}{
		\If{Find(U,v) $\neq$ Find(U,w)}{
			A = A $\cup$ {(v,w)}\;
			Union(U,v,w)
		}
	}
	return A\;
	\caption{Kruskal con Union-Find}
\end{algorithm}
\subsection{Struttura del progetto}
Nelle successive sezioni descriveremo in dettaglio l'implementazione di questi algoritmi e delle relative strutture dati di supporto e li testeremo su un dataset generato randomicamente, analizzando i risultati ottenuti e le performance in termini di tempo d'esecuzione e spazio allocato in memoria.\\
I test saranno condotti su grafi non necessariamente semplici (quindi con la possibile presenza di \textit{self loops} e archi multipli tra due nodi) e con pesi non necessariamente positivi.\\
Abbiamo scelto di implementare tali algoritmi utilizzando il linguaggio \textit{C++17} per motivi di efficienza e per potere, allo stesso tempo, utilizzare astrazioni tipiche dei linguaggi ad oggetti rendendo il codice prodotto più modulare.
\section{Implementazione}
In questa sezione verranno esposte e adeguatamente motivate le scelte implementative adottate durante lo sviluppo. L'intero progetto è stato realizzato facendo il più possibile uso di codice generico (template di classe per le strutture dati di supporto e template di funzione per gli algoritmi).\\
Infine verrà data una spiegazione dettagliata sulla struttura del codice realizzato ed eventuali note per la compilazione.
\subsection{Parser}
\subsection{Strutture dati}
\subsubsection{MinHeap}
\subsubsection{Union-Find}
\subsection{Strutture per la rappresentazione di grafi}
\subsubsection{Edge}
Per rappresentare un lato composto dai  suoi due nodi adiacenti e il peso associato abbiamo creato la classe templetizzata \textbf{Edge$<$W$>$}, con
%TODO tipo del peso non mi piace troppo
\textbf{W} tipo del peso.\\
Nella classe abbiamo ridefinito l'operatore di relazione \textbf{$<$} in modo tale che confronti il peso tra due lati, dato che rappresentiamo l'insieme di lati del grafo \textbf{E} utilizzando un \verb|std::vector|$<$\textbf{Edge}$>$ per poter ordinare i lati in ordine crescente facendo uso della funzione \verb|std::sort()| che prende in input le posizioni estreme del vettore e utilizza l'operatore \textbf{$<$} per l'ordinamento.
\subsubsection{Adjacency List}
Per rappresentare il grafo abbiamo utilizzato una lista di adiacenza implementata dalla classe \textbf{AdjacencyList$<$W$>$}, con \textbf{W} tipo del peso dei lati.\\
La classe è composta da un \verb|std::vector| in cui ogni indice rappresenta un nodo, e ogni elemento è un \verb|std::vector| di coppie \verb|std::pair| nel quale il primo membro è a sua volta un intero che rappresenta un nodo adiacente all'indice, mentre il secondo elemento è il peso associato al lato che collega i due nodi.\\
Nella classe abbiamo implementato \verb|total_weight()| per ottenere il peso totale del grafo, \verb|get_nodes()| e \verb|get_edges()| che ritornano il numero di nodi e di lati.\\
Per verificare se due nodi sono connessi nel grafo abbiamo implementato la funzione ricorsiva \verb|DFS(int v,int w)|:
\lstset{language=c++, style=mystyle}
\lstinputlisting[language=c++]{DFS.cpp}
La quale internamente chiama la funzione privata \verb|DFS_inside(int v, int w, bool L[])| la cui implementazione è la stessa di \verb|DFS| con l'unica differenza che il vettore \verb|L| non è creato ma ricevuto da \verb|DFS|.\\
In \verb|L| ogni indice rappresenta un nodo e ogni elemento indica se quel nodo è stato già visitato durante la ricerca.\\
Abbiamo implementato anche la funzione \verb|BFS(int v,int w)| in forma iterativa per confrontare la velocità tra i due algoritmi in \textbf{Kruskal naive}:
\lstset{language=c++, style=mystyle}
\lstinputlisting[language=c++]{BFS.cpp}
\subsection{Algoritmi}
\subsubsection{Prim}
\newpage
\subsubsection{Naive Kruskal}
\begin{flushleft}
La nostra implementazione presente nel file \verb|algorithms\kruskal_naive.h| è la seguente:

\lstset{language=c++, style=mystyle}
\lstinputlisting[language=c++]{k_naive.cpp}

\textbf{Note implementative:}

\medskip
Per l'ordinamento della lista di lati:

\lstset{language=c++, style=mystyle,backgroundcolor=\color{white}, firstnumber=2}  	 	
\begin{lstlisting}
sort edges of G by cost
\end{lstlisting}

\smallskip
si è utilizzata la funzione \verb|std::sort()| che assicura l'ordinamento in tempo O(non mi ricordo scrivi tu Enrico ammortizzato)%%TODO 
 è stato necessario ridefinire l'operatore booleano di minimo nell'oggetto EDGE
 
\lstset{language=c++, style=mystyle, firstnumber=3} 	 	
\begin{lstlisting}
std::sort(E.begin(), E.end());
\end{lstlisting}

\medskip
Per il controllo dell'aciclicità nel caso di aggiunta di un nuovo lato:

\lstset{language=c++, style=mystyle,backgroundcolor=\color{white}, firstnumber=4}  	 	
\begin{lstlisting}
if A U {e} is acyclic then
\end{lstlisting}

\smallskip
sono stati implementati sia l'algoritmo ricorsivo DFS che l'algoritmo iterativo BFS, queste due funzioni sono state ottimizzate in modo tale che ritornino TRUE solo se esiste un cammino tra i nodi adiacenti al lato che stiamo analizzando.

Se ritornano FALSE allora possiamo aggiungere il nuovo lato alla lista di adiacenza con la certezza che non verrà a formarsi un ciclo.

\lstset{language=c++, style=mystyle, firstnumber=5}
\begin{lstlisting}
if(!A.DFS(E[i].get_node_1(), E[i].get_node_2()))
\end{lstlisting}

\medskip
Inoltre si è ottimizato l'agoritmo con un controllo addizionale per far terminare anticipatamente l'iterazione. 

Viene controllato che il numero di lati non sia mai maggiore o uguale al numero di nodi del grafo.

\lstset{language=c++, style=mystyle, firstnumber=7}
\begin{lstlisting}
if(A.get_edges() == (A.get_nodes() - 1))
    break;
\end{lstlisting}
\end{flushleft}
\subsubsection{Kruskal (con Union-Find)}
\begin{flushleft}
La nostra implementazione presente nel file \verb|algorithms\kruskal_union_find.h| è la seguente:

\lstset{language=c++, style=mystyle}
\lstinputlisting[language=c++]{k_union.cpp}

\textbf{Note implementative:}

\medskip
Per l'inizializzazione della struttura Union-Find:

\lstset{language=c++, style=mystyle,backgroundcolor=\color{white}, firstnumber=2}  	 	
\begin{lstlisting}
U = initialize(V)
\end{lstlisting}

\smallskip
al costruttore è passato un vettore di interi con valori distinti per far si che inizialmente tutti i nodi appartengano a componenti connesse distinte.

\lstset{language=c++, style=mystyle, firstnumber=3}  	 	
\begin{lstlisting}
int *V;
V = new int[n_vec];
std::iota(V, V + n, 0);
UnionFind U(V, n_vec);
\end{lstlisting}

\medskip
Per l'ordinamento della lista di lati:

\lstset{language=c++, style=mystyle,backgroundcolor=\color{white}, firstnumber=2}  	 	
\begin{lstlisting}
sort edges of G by cost
\end{lstlisting}

\smallskip
si è utilizzata la funzione \verb|std::sort()| che assicura l'ordinamento in tempo O(non mi ricordo scrivi tu Enrico ammortizzato)%%TODO 
 è stato necessario ridefinire l'operatore booleano di minimo nell'oggetto EDGE
 
\lstset{language=c++, style=mystyle, firstnumber=7} 	 	
\begin{lstlisting}
std::sort(E.begin(), E.end());
\end{lstlisting}

\medskip
Per controllare se due nodi appartengono alla stessa componente connessa e per unire le componenti connesse
\lstset{language=c++, style=mystyle,backgroundcolor=\color{white}, firstnumber=2}  	 	
\begin{lstlisting}
if Find(U,v) =/= Find(U,w) then
	A = A U (v,w)
	Union(U,v,w)
\end{lstlisting}

\smallskip
sono state implementate le funzioni \verb|U.find()| e \verb|U.unite|.

\lstset{language=c++, style=mystyle, firstnumber=9} 	 	
\begin{lstlisting}
if (U.find(e.get_node_1()) != U.find(e.get_node_2())) {
	A.add(e);
    U.unite(e.get_node_1(), e.get_node_2());
}
\end{lstlisting}

\medskip
Inoltre si è ottimizato l'agoritmo con un controllo addizionale per far terminare anticipatamente l'iterazione. 

Viene controllato che il numero di lati non sia mai maggiore o uguale al numero di nodi del grafo.

\lstset{language=c++, style=mystyle, firstnumber=13}
\begin{lstlisting}
if(A.get_edges() == (A.get_nodes() - 1))
    break;
\end{lstlisting}
\end{flushleft}
\subsection{Note su compilazione e struttura del codice}
\textit{/*forse spostare in sezione a parte?*/}
\section{Testing e analisi sperimentali}
\subsection{Risultati prodotti}
[...]
\section{Conclusioni}
\end{document}          
